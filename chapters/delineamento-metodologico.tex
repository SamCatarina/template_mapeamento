\chapter{Delineamento Metodológico}\label{delineamento}

Este estudo adotou uma abordagem quantitativa para investigar o fenômeno da disparidade de gênero em áreas de STEM. A pesquisa focou na identificação de padrões e na construção de perfis de estudantes com base em suas escolhas. Trata-se de um estudo transversal, o que significa que a coleta de dados ocorreu em um único período de tempo, oferecendo um panorama das variáveis em questão no momento da pesquisa.

A população-alvo desta pesquisa foi composta por estudantes do ensino superior, matriculados em diferentes áreas do conhecimento (incluindo STEM e não-STEM), em instituições públicas e privadas localizadas na cidade de Salgueiro-PE. A amostra foi definida por conveniência, buscando a participação voluntária dos alunos. Todos os participantes foram informados sobre os objetivos do estudo e tiveram sua anonimidade e confidencialidade asseguradas, em conformidade com a LGPD \cite{brasil_lgpd_2018}.

Os dados foram coletados por meio de um questionário estruturado e autoadministrado, aplicado de forma online via plataforma autoral disponibilizada no site \footnote{Site disponível em: \url{https://samcatarina.github.io/MulheresNaCiencia/}}. O instrumento foi composto majoritariamente por questões objetivas e fechadas, organizadas em quatro eixos temáticos principais, identificados por meio literatura: influências familiares, influências educacionais, fatores psicológicos e características socioeconômicas e demográficas. Essas seções foram projetadas para coletar informações detalhadas que abordassem os fatores que influenciaram as escolhas de curso (sejam eles na área STEM ou não-STEM), além de outras variáveis relevantes para a formação dos perfis dos estudantes. Houve questões abertas apenas nos casos em que foi necessário compreender o conceito ou a forma de pensamento dos participantes sobre determinado assunto.

%pre-proc

Antes da análise, os dados coletados passaram por um processo de pré-processamento, essencial para garantir a qualidade e a consistência dos resultados. Inicialmente, foi realizada uma inspeção para identificar e tratar valores ausentes, inconsistentes ou duplicados. As variáveis numéricas foram normalizadas utilizando a técnica de normalização min-max \cite{han2011datamining}, a fim de padronizar a escala dos dados entre 0 e 1, prevenindo que variáveis com escalas maiores dominassem os algoritmos de agrupamento.

Grande parte das respostas às questões fechadas seguiu uma escala ordinal de intensidade ou frequência baseadas da escala Linkert \cite{costa_likert_2024}, o que permitiu sua transformação em valores numéricos com base na ordem natural das alternativas. Essa conversão preservou a semântica da resposta, permitindo que tais variáveis fossem utilizadas diretamente nos métodos quantitativos, como o agrupamento.

%analise-estatistica

Após o pré-processamento, foi conduzida uma análise estatística descritiva para caracterizar o perfil geral da amostra. Foram calculadas frequências absolutas e relativas, medidas de tendência central (como média e mediana) e de dispersão (como desvio padrão), conforme o tipo de variável. Os dados foram apresentados por meio de tabelas e gráficos descritivos, incluindo gráficos de barras que facilitaram a visualização e interpretação das principais características dos participantes.

Essa etapa visou identificar tendências gerais da amostra, como a proporção de estudantes em áreas STEM e não-STEM e os níveis de interesse por disciplinas de exatas e tecnologia. Também foram observadas correlações iniciais entre as variáveis, com o intuito de levantar hipóteses para as etapas posteriores de análise.

%agrupamento

Para a construção dos perfis de estudantes, foram aplicadas técnicas de agrupamento (clustering), com o objetivo de identificar grupos com características semelhantes em relação às variáveis investigadas. A técnica escolhida para essa análise foi a clusterização hierárquica \cite{han2011datamining}, utilizando o método de Ward, que minimiza a variância intra-cluster. A distância entre os pontos foi calculada utilizando a métrica de distância euclidiana, adequada para variáveis numéricas. Além disso, as features selecionadas para a montagem dos clusters foram definidas na fase de análise de importância das variáveis, utilizando o modelo Random Forest junto com os valores SHAP.

A definição do número ideal de grupos foi feita com base no método do cotovelo (elbow method). Após a criação dos clusters, cada grupo foi analisado em termos de suas características predominantes, permitindo a descrição dos perfis de estudantes identificados.
