% ----------------------------------------------------------
\chapter{Conclusões}\label{conclusoes}
% ----------------------------------------------------------

Este trabalho partiu da observação de um padrão de distribuição de gênero nas diferentes áreas do conhecimento: embora as mulheres sejam maioria no Ensino Superior brasileiro, sua presença permanece concentrada em determinadas áreas, com sub-representação nas carreiras de STEM. A motivação central foi compreender os fatores que influenciam a escolha de curso superior e identificar perfis de estudantes capazes de explicar, parcial ou totalmente, essa desigualdade de representatividade. Buscou-se, com isso, fornecer subsídios empíricos que possam orientar políticas e práticas de incentivo ao ingresso e à permanência das mulheres em áreas científicas e tecnológicas.

Em alinhamento a essa motivação, os objetivos propostos foram atendidos de forma parcial e coerente com o escopo da pesquisa: elaborou-se e aplicou-se um instrumento de coleta de dados em instituições de ensino superior da região do Sertão Central de Pernambuco; realizou-se análise exploratória dos dados; treinou-se um modelo supervisionado explicável (Random Forest + interpretação por valores SHAP) para identificar variáveis associadas a diferenças de gênero e interesse por STEM; e empregou-se uma técnica de clusterização hierárquica (método de Ward) para mapear perfis de estudantes a partir das principais variáveis identificadas.

Metodologicamente, a pesquisa combinou técnicas quantitativas clássicas e modernas: processamento e codificação das respostas do questionário, análise exploratória, modelagem supervisionada com Random Forest e interpretação de importância e direção de efeito via SHAP, seguida de clusterização hierárquica com validação visual pelo dendrograma e método do cotovelo. Essa combinação permitiu não apenas identificar quais variáveis se associam com maior força às diferenças por gênero, mas também descrever agrupamentos de respondentes com perfis similares.

Os principais achados podem ser sumarizados do seguinte modo:

- A percepção de exclusão em ambientes de tecnologia (\textit{feelingExcludedTech}) emergiu como a variável mais relevante para distinguir perfis de gênero. Valores mais altos dessa variável estão fortemente associados ao público feminino, apontando para um problema de percepção e pertencimento que pode limitar a intenção de seguir carreiras tecnológicas.
- A ocorrência de restrição de atividades por motivo de gênero e a concordância com estereótipos sobre profissões também se mostraram importantes para diferenciação entre os grupos, indicando que normas e representações sociais continuam a influenciar escolhas e autopercepções.
- Incentivo familiar para os estudos e interesse prévio por disciplinas de exatas apareceram como fatores positivos associados ao maior interesse em seguir carreira em STEM.
- A análise por cluster revelou dois perfis principais: o Cluster 1 (predominantemente masculino na amostra) caracterizado por menor identificação com STEM, maior concordância com estereótipos e menor incentivo familiar; e o Cluster 2 (mais heterogêneo) com maior interesse prévio em exatas, maior incentivo familiar e menor internalização de estereótipos.

Apesar das contribuições, a pesquisa tem limitações relevantes que merecem ser destacadas. A amostra é relativamente pequena (n = 88), com distribuição de gênero e composição institucional que limitam a generalização dos resultados para outras regiões ou para populações maiores. Algumas variáveis apresentaram baixa variabilidade dentro de subgrupos (por exemplo, sentimento de exclusão muito presente entre as mulheres), o que reduz sua capacidade discriminatória em análises estratificadas. Finalmente, escolhas de codificação e normalização de categorias, necessárias para a modelagem, podem inserir arbitrariedades que influenciam interpretações numéricas e comparações.

Em termos de contribuição, o estudo agrega: (i) um levantamento empírico localizado sobre percepções de gênero e interesse em STEM no município de Salgueiro–PE; (ii) a aplicação integrada de modelos explicáveis (SHAP) e técnicas de clusterização para mapear perfis de estudantes, ampliando a compreensão sobre como fatores psicossociais e contextuais se organizam; e (iii) o desenvolvimento (conforme proposto nas seções anteriores) de uma plataforma de divulgação de pesquisadoras brasileiras, que busca aumentar a visibilidade de modelos de referência e reduzir barreiras de representação.




\section{Trabalhos futuros}

Para trabalhos futuros, recomenda-se: ampliar a amostra e diversificar a abrangência geográfica para aumentar a robustez e a generalização dos achados; realizar desenhos longitudinais que permitam observar trajetórias e efeitos temporais (por exemplo, observando se intervenções de mentoria modificam intenção e permanência em STEM); incorporar métodos mistos, incluindo entrevistas qualitativas com estudantes e familiares, para aprofundar a compreensão das causas subjacentes aos sentimentos de exclusão e às dinâmicas de incentivo familiar; testar intervenções experimentais (programas de mentoria, oficinas com modelos femininos em STEM, mudanças curriculares) e avaliar seu impacto; e explorar outras técnicas de agrupamento e validação (por exemplo, LPA, clustering baseado em mistura ou validação em amostras externas).

Conclui-se que, embora não exista solução única para a sub-representação feminina em STEM, a combinação de políticas institucionais (mais modelos de referência, práticas pedagógicas acolhedoras), ações familiares e intervenções locais informadas por evidência empírica pode contribuir para reduzir barreiras percebidas e ampliar o sentido de pertencimento. Espera-se que os resultados aqui apresentados sirvam como ponto de partida para ações contextuais no Sertão Central de Pernambuco e como base para pesquisas ampliadas sobre o tema.
