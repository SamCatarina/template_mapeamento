
\chapter{Fundamentação Teórica}\label{fundamentacao}

\section{Determinantes Psicossociais na Escolha de Carreira}

\subsection{Estereótipos de Gênero e sua Influência nas Escolhas Acadêmicas}
Em primeiro lugar, é preciso estarmos cientes sobre o real cenário de participação feminina na graduação brasileira. Segundo dados do censo de educação superior, mulheres representam a maioria dos estudantes brasileiros de graduação e pós-graduação \cite{inep2023}.
Porém, ao filtrarmos por áreas, percebemos que as mulheres se concentram em áreas específicas do conhecimento, como, por exemplo, relacionadas à saúde e educação (cursos de licenciatura). Ou seja, o cenário real do Brasil é que existem muitas mulheres na graduação, porém estão concentradas em determinadas áreas.

Se o problema não é a quantidade de mulheres na graduação, precisamos analisar quais são os fatores que influenciam a escolha de cursos específicos pelas mulheres.

Segundo um estudo realizado por \cite{master2021}, é evidente a influência dos estereótipos de gênero sobre o interesse e as habilidades atribuídas a meninas e meninos desde a infância. Observou-se que, já na educação infantil, as crianças tendem a reproduzir papéis de gênero nas diferentes áreas do conhecimento. Em especial, foi identificado que elas associam as áreas de STEM (Ciência, Tecnologia, Engenharia e Matemática) mais frequentemente aos homens, tanto em termos de interesse quanto de competência.

O estudo também diferencia os estereótipos de gênero em relação ao interesse e à habilidade: enquanto o interesse se refere à percepção de que determinadas áreas são mais atrativas ou adequadas para um gênero específico, a habilidade diz respeito à crença de que meninos ou meninas são naturalmente mais capazes ou competentes em certas disciplinas.

Um segundo estudo sobre a presença de estereótipos entre estudantes \cite{mcguire2022} revela que os estereótipos de gênero relacionados às áreas de STEM estão presentes tanto em crianças quanto em adolescentes, do ensino fundamental ao médio. A manifestação desses estereótipos varia conforme o gênero dos estudantes: meninos tendem a reproduzi-los de forma mais acentuada, enquanto meninas os reproduzem em menor grau — embora ainda de maneira significativa.

O primeiro estudo também constatou que as meninas que reproduzem estereótipos em relação a STEM tendem a demonstrar menor interesse por atividades descritas de forma estereotipada. Ou seja, quanto mais as estudantes internalizam esses estereótipos, menos se identificam ou se engajam com determinadas tarefas que são socialmente associadas ao universo masculino.

A preferência por não se envolver em uma atividade descrita de forma estereotipada como masculina pode ser explicada pelo sentimento de pertencimento. Um estudo realizado com estudantes de 11 a 14 anos, em uma escola nos Estados Unidos \cite{opps2022}, revelou uma relação significativa entre o sentimento de pertencimento à área de ciência da computação e o nível de estereótipos que as alunas reproduzem. Observou-se que meninas que representam cientistas de forma estereotipada — especialmente com traços masculinos ou pejorativos — tendem a sentir-se menos pertencentes a essa área. Além disso, o estudo mostrou que, no que se refere aos estereótipos de aparência, as meninas demonstraram uma frequência significativamente maior do que os meninos ao retratar cientistas com características estereotipadas, como jaleco, óculos ou cabelos desalinhados.

\subsection{Fatores Socioeconômicos e Geográficos na Escolha Acadêmica}

De acordo com \cite{park2024}, estudantes de menor nível socioeconômico tendem a enfrentar maiores dificuldades para transformar aspirações elevadas em resultados concretos, pois carecem de recursos e suporte que possibilitem alcançar essas metas. Mesmo quando almejam ocupações de maior prestígio ou renda, suas aspirações nem sempre se convertem em trajetórias acadêmicas ou profissionais viáveis, refletindo limitações impostas pelo contexto econômico e social.

A compreensão do interesse em STEM entre estudantes do Sertão Central de Pernambuco exige considerar as condições socioeconômicas e, sobretudo, as oportunidades educacionais disponíveis na região.

De acordo com o Censo da Educação Superior de 2023, o município de Salgueiro oferece 17 cursos superiores presenciais. Desses, cerca de 7 estão diretamente relacionados às áreas de STEM (Ciência, Tecnologia, Engenharia e Matemática). Além disso, 8 cursos têm como foco a formação de professores, sendo ofertados na modalidade de licenciatura. Vale destacar que apenas duas opções de cursos STEM com grau de bacharelado são ofertadas por instituições públicas, o que pode representar uma limitação no acesso a essas áreas estratégicas para o desenvolvimento científico e tecnológico da região. Conforme o próprio plano pedagógico do curso de Ciência da \cite{univasf2021ppc}, um dos objetivos de criação do curso é justamente atender às demandas técnicas e tecnológicas atuais e futuras do Sertão Central, contribuindo para o desenvolvimento regional.


\subsection{Fatores Familiares na Escolha Acadêmica}

Na revisão bibliográfica realizada por \cite{gencel2025}, a influência parental foi identificada como um dos principais fatores que contribuem para a baixa representatividade de mulheres hispânicas nas áreas de STEM (Ciência, Tecnologia, Engenharia e Matemática). Observou-se que estudantes que recebem pouco incentivo ou reconhecimento por parte dos pais tendem a desenvolver uma percepção negativa sobre suas próprias habilidades em disciplinas como ciências e matemática, o que pode comprometer seu interesse e desempenho nessas áreas.

Esses dados evidenciam a relevância do papel da família no processo de escolha da carreira acadêmica. Quanto menor o incentivo recebido em determinadas áreas do conhecimento, menor tende a ser a autoconfiança dos estudantes para seguir uma trajetória profissional nessas áreas, uma vez que há uma maior propensão a acreditarem que suas capacidades são inferiores às exigidas.

Além da falta de apoio em determinadas áreas, há também a influência significativa das expectativas familiares sobre o futuro dos estudantes. Em uma revisão bibliográfica realizada por \cite{arshad2024}, foi evidenciado que, em muitas culturas e contextos sociais, as decisões da família tendem a ter um peso maior do que os próprios desejos do estudante. Em outras palavras, a vontade familiar, em diversos casos, sobrepõe-se às aspirações individuais dos jovens, influenciando diretamente suas escolhas acadêmicas e profissionais.

\subsection{Fatores Educacionais na Escolha Acadêmica}

Segundo um panorama levantado por \cite{pugliese} sobre o estado da educação em STEM no Brasil, foi notado que ainda se trata de um movimento incipiente no país, com presença tímida na literatura acadêmica nacional e maior disseminação em escolas privadas e iniciativas de organizações não governamentais. Além disso, observa-se que, quando presente, o modelo é muitas vezes adotado como tendência estrangeira, com forte apelo de mercado, e não como uma proposta pedagógica adaptada às realidades e necessidades locais.

Esses fatores acabam afastando o interesse dos alunos, especialmente aqueles da rede pública, uma vez que o conteúdo frequentemente não dialoga com seu contexto social e cultural, nem considera as desigualdades estruturais que impactam seu acesso ao conhecimento. Sem estratégias inclusivas e contextualizadas, o ensino de STEM corre o risco de se tornar excludente, reforçando barreiras já existentes ao invés de superá-las.

Além disso, considerando os achados de \cite{morales2021} sobre o impacto da discordância mentor-mentorado na intenção de estudantes latinos de buscar a pós-graduação e na sua produtividade em pesquisa, fica evidente a crucial importância do mentor na trajetória acadêmica e profissional dos mentorados. O estudo revela que, enquanto a discordância de gênero pode, surpreendentemente, estar associada a um aumento de (17\%) na intenção de pós-graduação para os estudantes latinos em geral, há uma nuance crítica: quando pareadas com mentores de gênero discordante, estudantes Latinas foram (70\%) menos propensas a apresentar seus projetos de pesquisa em conferências profissionais. Isso sublinha que, embora a mentoria com diversidade de gênero possa, em certos aspectos, impulsionar as aspirações de longo prazo, ela pode, ao mesmo tempo, criar barreiras significativas para a participação ativa em marcos de produtividade de pesquisa de curto prazo. Complementarmente, a discordância de raça/etnia e, mais acentuadamente, a de status de primeira geração, está ligada a uma redução significativa da intenção de buscar a pós-graduação. Assim, a relação de mentoria não é neutra; a similaridade de experiências e backgrounds, especialmente em dimensões sociais, raciais e de status familiar no ensino superior, pode ser um fator facilitador ou, na sua ausência (discordância), um obstáculo. 


\section{Técnicas de Agrupamento como Estratégia de Mapeamento de Perfis}

O agrupamento, ou análise de \textit{clusters}, é uma técnica fundamental da mineração de dados voltada para a descoberta de padrões e estruturas ocultas em conjuntos de dados. Trata-se do processo de organizar objetos em grupos de tal forma que os elementos pertencentes ao mesmo grupo sejam altamente semelhantes entre si, enquanto apresentem grande dissimilaridade em relação aos elementos de outros grupos. Segundo \cite{han2011datamining}, a avaliação dessas similaridades ou dissimilaridades ocorre com base nos atributos dos objetos, sendo comum o uso de medidas de distância como critério quantitativo para definir a proximidade entre os dados.

O agrupamento se diferencia de outras técnicas de aprendizado de máquina por ser uma abordagem de aprendizado não supervisionado. Isso significa que, ao contrário de métodos supervisionados como classificação, em que os dados de entrada estão associados a rótulos de classe previamente definidos, o agrupamento opera sem qualquer informação prévia sobre categorias. Como afirmam os autores \cite{han2011datamining}, o agrupamento é, portanto, uma forma de "aprendizado por observação", sendo especialmente útil em contextos nos quais não se conhece previamente a estrutura dos dados ou os agrupamentos naturais existentes.

A aplicação da análise de agrupamento é particularmente relevante no contexto da educação para a criação de perfis de estudantes. Um exemplo claro dessa aplicação é o trabalho de \cite{oliveira2022}. Neste estudo, os autores realizam uma revisão da literatura e uma análise de diferentes algoritmos de clusterização com o objetivo de identificar e compreender padrões de engajamento de estudantes em ambientes de aprendizagem online. Ao agrupar estudantes com comportamentos de engajamento semelhantes, é possível traçar perfis específicos que revelam, por exemplo, alunos altamente ativos, moderadamente engajados ou aqueles com baixo nível de participação. 

Expandindo essa aplicação para um contexto de saúde e bem-estar, \cite{martins2021} demonstraram o potencial do agrupamento na identificação de perfis de risco. O objetivo do trabalho foi agrupar adolescentes escolares com características semelhantes em relação à predisposição ao uso de substâncias psicoativas. Ao aplicar técnicas de clusterização sobre dados de variáveis sociodemográficas, comportamentais e relacionadas à saúde, os pesquisadores conseguem segmentar a população estudada em perfis distintos, como aqueles com maior vulnerabilidade devido a fatores familiares, sociais ou psicológicos, e aqueles com menor risco. 


De acordo com \cite{han2011datamining}, a clusterização hierárquica é uma técnica de agrupamento que organiza os objetos de um conjunto de dados em uma estrutura de níveis, formando uma hierarquia de clusters. Diferentemente dos métodos particionais, que particionam os dados diretamente em um número pré-determinado de grupos, a clusterização hierárquica constrói uma representação em forma de árvore que evidencia como os objetos se agrupam ou se separam ao longo das etapas do processo. Essa representação gráfica é denominada dendrograma, o qual fornece uma visualização clara dos relacionamentos entre os dados em diferentes níveis de granularidade, permitindo compreender tanto a formação de grupos mais amplos quanto suas subdivisões internas.

A construção dessa hierarquia pode ocorrer de duas maneiras, que definem os principais tipos de métodos hierárquicos: aglomerativo e divisivo.

O método aglomerativo, também conhecido como abordagem bottom-up, inicia considerando cada objeto como um cluster isolado. Em seguida, os clusters mais semelhantes são iterativamente fundidos, resultando em grupos progressivamente maiores. Esse processo continua até que todos os objetos estejam reunidos em um único cluster ou até que um critério de parada seja alcançado. O dendrograma desse tipo de método registra cada fusão e sua ordem, permitindo observar como os grupos emergem e se relacionam ao longo do processo. Um aspecto característico dessa abordagem é sua irreversibilidade: uma vez realizada a fusão entre dois clusters, essa decisão não pode ser revertida.

O método divisivo, ou abordagem top-down, segue a lógica oposta. O algoritmo inicia com todos os objetos reunidos em um único cluster e procede realizando divisões sucessivas, segmentando o conjunto em clusters menores. As divisões continuam de maneira recursiva até que os clusters atendam a um nível desejado de homogeneidade ou até que reste apenas um objeto por grupo. Assim como no método aglomerativo, o processo registrado no dendrograma mostra a ordem e o nível em que cada divisão ocorre, embora essa abordagem tipicamente demande maior custo computacional.

Em ambos os casos, a utilização do dendrograma como representação permite visualizar a estrutura hierárquica dos dados, identificar níveis de similaridade e determinar, de forma interpretável, quantos clusters são mais adequados à análise. \cite{han2011datamining}

Ainda de acordo com \cite{han2011datamining}, os métodos hierárquicos aglomerativos utilizam medidas de distância ou similaridade para determinar quais grupos devem ser fundidos em cada etapa. A seguir apresentam‑se, de forma descritiva e sem fórmulas, as quatro medidas clássicas de ligação (\emph{linkage}) entre clusters, o que é suficiente para contextualizar a escolha do método adotado neste trabalho.

    	\textbf{Single linkage (mínima distância)}

    Nesse procedimento, a ligação entre dois grupos é determinada pela menor distância observada entre quaisquer dois pontos pertencentes a esses grupos. Por considerar apenas o par de pontos mais próximos, o método favorece a formação de cadeias e estruturas alongadas, sendo sensível a ruídos e pontos isolados que podem ligar clusters distintos.

    	\textbf{Complete linkage (máxima distância)}

    Aqui a fusão é conduzida pela maior distância entre pares de observações dos dois grupos. Esse critério tende a gerar clusters mais compactos e homogêneos, já que exige que todos os pontos do grupo resultante fiquem relativamente próximos; por outro lado, pode ser influenciado por outliers ao forçar partições mais restritas.

    	\textbf{Centroid linkage (distância entre centróides)}

    O critério baseia‑se na distância entre os centróides (médias) dos clusters. É uma abordagem simples e intuitiva, porém pode provocar inversões (\emph{reversals}) no dendrograma em determinadas configurações, ou seja, a ordem de fusões pode não refletir monotonicamente a proximidade original entre elementos.

    	\textbf{Average linkage (distância média)}

    Neste método, a ligação é calculada a partir da média das distâncias entre todos os pares de pontos pertencentes aos dois grupos. Funciona como um compromisso entre os extremos representados pelo single e pelo complete linkage, apresentando maior robustez a ruídos e formando clusters com grau intermediário de compactação.

    \textbf{Método de Ward (mínima variância)}

Em contraste com esses critérios baseados em distâncias diretas, o método de Ward utiliza outro princípio para decidir a fusão de clusters. Segundo \cite{randriamihamison2021_hac}, o método adota uma perspectiva de minimização da variância interna, avaliando, a cada etapa, qual fusão provoca o menor aumento na soma das dispersões intracluster (within-cluster sum of squares). Assim, em vez de observar apenas pares específicos de pontos ou a média entre eles, o algoritmo considera o efeito global que uma fusão teria na homogeneidade e na compactação dos grupos.

Enquanto os métodos clássicos podem favorecer estruturas alongadas (single linkage), muito compactas (complete linkage) ou instáveis (centroid linkage), o método de Ward produz clusters mais equilibrados, coesos e com baixa variabilidade interna, justamente por otimizar continuamente a qualidade da partição. Como destacam \cite{randriamihamison2021_hac}, essa característica confere ao método maior estabilidade e interpretabilidade, além de dendrogramas mais consistentes, especialmente em conjuntos de dados onde a homogeneidade dos grupos é um objetivo central.

Em síntese, o método de Ward difere fundamentalmente dos demais porque não se apoia apenas em medições de distância entre objetos ou centroides, mas em um critério estatístico de compacidade mínima, funcionando como um processo de otimização que privilegia a formação de clusters internamente homogêneos.

A determinação do número adequado de clusters é uma etapa fundamental em algoritmos de agrupamento, especialmente naqueles que exigem a definição prévia desse parâmetro. Entre os métodos mais utilizados para essa finalidade, destaca-se o Método do Cotovelo (Elbow Method), cuja popularidade se deve à sua simplicidade e caráter visual.

Como descrito por \cite{shi2021_elbow}, o Método do Cotovelo baseia-se em observar como a distorção interna dos clusters — isto é, o grau de compactação dos grupos — se comporta à medida que o número de clusters aumenta. Em geral, adicionar mais clusters tende a tornar cada grupo mais homogêneo, reduzindo o erro interno. No entanto, essa redução não é constante: após certo ponto, criar novos clusters deixa de gerar uma melhoria significativa.

O Método do Cotovelo explora exatamente esse comportamento. Ele consiste em executar o algoritmo de clustering para diferentes valores de k e, em seguida, construir um gráfico que relaciona o número de clusters com a medida de erro. A interpretação visual busca identificar um ponto de inflexão — o “cotovelo” — onde a curva deixa de apresentar quedas acentuadas e passa a diminuir mais lentamente. Esse ponto representa um equilíbrio entre ganho de qualidade e simplicidade do modelo, sendo interpretado como o número apropriado de clusters para a estrutura dos dados.

\cite{shi2021_elbow} ressaltam, porém, que essa identificação depende da clareza do formato da curva. Quando o gráfico apresenta um cotovelo evidente, a escolha é relativamente simples; entretanto, quando a curva é muito suave, a identificação do ponto ideal torna-se subjetiva e pode variar entre analistas. Essa limitação motivou os autores a propor um método quantitativo alternativo para detectar automaticamente o cotovelo, evidenciando que, apesar de amplamente utilizado, o Método do Cotovelo nem sempre fornece uma indicação precisa em situações de baixa contrastividade na curva.

Em síntese, o Método do Cotovelo é uma ferramenta intuitiva e amplamente empregada para estimar o número de clusters, fundamentando-se na análise visual da redução do erro. No entanto, como discutem \cite{shi2021_elbow}, sua natureza subjetiva pode limitar sua eficácia em cenários em que o formato da curva não apresenta um ponto de inflexão claramente definido.



\section{Explicabilidade e Interpretabilidade em Modelos de AM}

1. Conceitos de Explainable Artificial Intelligence (XAI)
Explainable Artificial Intelligence (XAI) refere-se à capacidade de sistemas de inteligência artificial de fornecer explicações claras, compreensíveis e significativas sobre seus processos de decisão. Segundo o relatório \textit{EDPS TechDispatch on Explainable Artificial Intelligence} \cite{edps_xai}, muitos modelos modernos, especialmente aqueles baseados em aprendizagem profunda, operam como verdadeiras “caixas-pretas”, dificultando a compreensão de sua lógica interna tanto por usuários quanto pelos próprios engenheiros responsáveis. Essa opacidade pode ocultar vieses, erros ou correlações espúrias, gerando riscos significativos para indivíduos afetados por decisões automatizadas. Nesse contexto, o XAI busca tornar o comportamento dos modelos mais acessível ao ser humano, promovendo transparência, interpretabilidade e accountability (responsabilização e capacidade de prestar contas pelos resultados e decisões produzidos pelo modelo). O documento destaca que a explicabilidade deve permitir compreender competências do sistema, justificar decisões específicas e revelar informações relevantes sobre o processo decisório.

Os princípios de transparência, interpretabilidade e explicabilidade constituem elementos centrais no desenvolvimento de sistemas de Inteligência Artificial responsáveis. Conforme destaca o relatório \textit{EDPS TechDispatch on Explainable Artificial Intelligence} \cite{edps_xai}, a transparência refere-se à capacidade de compreender o funcionamento geral do sistema, sua finalidade, seus limites e as condições sob quais suas decisões são produzidas, permitindo que usuários e autoridades saibam ``o que o sistema faz'' e ``como o faz''. A interpretabilidade, por sua vez, diz respeito ao grau em que seres humanos conseguem entender as relações entre entradas, processamento interno e saídas do modelo, reduzindo o efeito de ``caixa-preta'' associado a abordagens opacas de aprendizado de máquina. Já a explicabilidade envolve fornecer justificativas claras, significativas e contextualizadas para decisões específicas tomadas pelo sistema, revelando as razões e fatores que contribuíram para determinado resultado. 


2. SHAP como métrica de contribuição e interpretabilidade
\\
Os valores SHAP (SHapley Additive exPlanations) constituem um método de explicabilidade baseado na teoria dos valores de Shapley, permitindo quantificar a contribuição individual de cada variável para a predição de um modelo. Conforme apresentado por \textit{nanal2024\_shap}, a previsão do modelo para a instância $i$ pode ser decomposta como a soma aditiva das contribuições de cada variável, expressa pela Equação (2): 
\[
\hat{y}_i = \text{shap}_0 + \text{shap}(X_{1i}) + \text{shap}(X_{2i}) + \cdots + \text{shap}(X_{ji}),
\]
na qual $\hat{y}_i$ representa a predição do modelo para o \textit{catchment} $i$, enquanto $\text{shap}(X_{ji})$ corresponde ao valor SHAP associado à $j$-ésima variável dessa instância. O termo $\text{shap}_0$ é definido na Equação (3),
\[
\text{shap}_0 = E(\hat{y}_i),
\]
sendo a média global das predições em todos os \textit{catchments}. Assim, como descreve explicitamente o artigo, os valores SHAP permitem interpretar a saída do modelo ao decompor sua predição em efeitos individuais atribuídos a cada variável, fornecendo transparência e suporte à interpretabilidade no contexto de Explainable Artificial Intelligence (XAI).



3. Uso de XAI para entender vieses de gênero, justiça algorítmica e tomadas de decisão baseadas em dados.

4. Explicar Random Forest e seu uso para detectar as variáveis importantes com SHAP

O algoritmo \textit{Random Forest}, proposto por \cite{breiman2001_random_forests}, é um método de aprendizado supervisionado baseado na combinação de múltiplas árvores de decisão. Em vez de treinar uma única árvore, o modelo constrói centenas ou milhares delas, cada uma gerada a partir de uma amostra aleatória do conjunto de dados original, obtida pelo método de \textit{bootstrap}. Esse procedimento consiste em selecionar exemplos aleatoriamente com reposição, de modo que cada árvore é treinada com um subconjunto ligeiramente diferente dos dados. Além disso, em cada divisão interna da árvore, apenas um subconjunto aleatório das variáveis é disponibilizado para escolha da melhor divisão. Essa dupla aleatoriedade — nos exemplos e nos atributos — reduz a correlação entre as árvores, aumenta a diversidade da floresta e torna o modelo mais robusto, estável e resistente ao \textit{overfitting} (situação em que o modelo aprende padrões específicos e ruídos do conjunto de treino, perdendo capacidade de generalização para novos dados), mesmo quando muitas árvores são utilizadas.

Uma característica importante do Random Forest é o uso das chamadas amostras \textit{out-of-bag} (OOB). Durante a criação de cada árvore, aproximadamente um terço das observações não é selecionado na amostra \textit{bootstrap}. Esses exemplos que ``ficam de fora'' são chamados de OOB e funcionam como uma espécie de conjunto de validação interno. Assim, cada árvore dispõe de exemplos que não foram usados em seu treinamento e que permitem avaliar seu desempenho sem necessidade de um conjunto externo de teste. Esse mecanismo interno de validação é central para o cálculo da importância das variáveis.

\cite{breiman2001_random_forests} descreve um procedimento simples e eficiente para medir a importância das variáveis (\textit{feature importance}) usando diretamente as estimativas OOB. Depois que cada árvore é construída, os valores da variável $m$ nos exemplos OOB são aleatoriamente permutados, isto é, embaralhados, produzindo uma versão ``deturpada'' dessa variável. Em seguida, esses exemplos modificados são passados novamente pela árvore, e a classificação obtida é comparada com aquela gerada quando os valores originais estavam intactos. Esse processo é repetido para cada variável, uma de cada vez. Ao final da construção de toda a floresta, compara-se a taxa de erro obtida com os dados originais com a taxa obtida quando cada variável foi artificialmente embaralhada. Segundo o autor, a importância de uma variável é dada pelo aumento percentual da taxa de erro OOB causado por essa permutação. Quanto maior o aumento no erro ao destruir a informação contida na variável, mais relevante ela é considerada para a predição do modelo.


Embora o Random Forest forneça uma estimativa interna de importância das variáveis por meio do aumento percentual do erro \textit{out-of-bag} (OOB), esse mecanismo apresenta uma visão essencialmente global e agregada da relevância dos atributos. Para aprofundar a compreensão sobre como cada variável influencia as predições, tanto no nível global quanto no nível individual, é possível combinar o modelo Random Forest com a metodologia SHAP, potencializando a interpretabilidade. Conforme discutido por \cite{nanal2024_shap} e fundamentado na teoria dos valores de Shapley, os valores SHAP quantificam a contribuição marginal de cada variável para a predição específica de uma instância, permitindo decompor a saída do modelo de forma aditiva e interpretável.

No contexto deste trabalho, após o treinamento do modelo Random Forest, os valores SHAP são calculados utilizando o TreeExplainer, um método de explicação otimizado especificamente para modelos baseados em árvores. Proposto por \cite{lundberg2020_treeexplainer}, o TreeExplainer constitui um avanço significativo na interpretabilidade desses modelos ao permitir o cálculo exato de valores de Shapley em tempo polinomial, explorando diretamente a estrutura hierárquica das árvores de decisão que compõem o algoritmo.

Enquanto o cálculo tradicional de valores de Shapley exige considerar todas as possíveis combinações de atributos — um problema classicamente NP-difícil, inviável para modelos reais — o TreeExplainer reformula esse processo ao colapsar o somatório combinatorial em operações estruturadas sobre os caminhos e folhas das árvores. Dessa forma, o método substitui aproximações estocásticas por um algoritmo determinístico capaz de computar explicações exatas, consistentes e localmente fiéis, preservando integralmente as propriedades fundamentais dos valores de Shapley, como eficiência, simetria e monotonicidade.

Essa eficiência decorre da capacidade do algoritmo de rastrear, ao longo dos caminhos decisórios, a proporção de subconjuntos de atributos que fluem para cada folha, o que permite atribuir corretamente a contribuição marginal de cada variável para a predição. \cite{lundberg2020_treeexplainer} demonstram que, ao explorar essa estrutura, o TreeExplainer supera limitações de abordagens anteriores, como métodos heurísticos específicos para árvores (por exemplo, Saabas), que sofrem de inconsistência e distorcem a importância de variáveis conforme sua profundidade na árvore, e métodos modelo-agnósticos baseados em amostragem, que apresentam alta variância e custos computacionais elevados 

Assim, o TreeExplainer oferece explicações precisas mesmo em modelos compostos por centenas de árvores, como florestas aleatórias, tornando possível interpretar de forma confiável a contribuição individual de cada atributo para cada previsão.

Além disso, enquanto as medidas tradicionais de importância do Random Forest fornecem apenas uma visão global das variáveis mais relevantes — frequentemente influenciada por métricas internas como ganho ou Gini — os valores SHAP calculados pelo TreeExplainer permitem compreender não apenas a magnitude, mas também o sentido (positivo ou negativo) e até interações locais entre variáveis em cada instância analisada. \cite{lundberg2020_treeexplainer} mostram que essa abordagem produz explicações que refletem com maior fidelidade o comportamento real do modelo e revelam padrões que métodos globais não conseguem capturar, como efeitos raros, interações específicas e subgrupos de dados com relações particulares entre atributos e predições 

Dessa forma, a combinação entre a importância global tradicional e os valores SHAP — calculados de forma eficiente e consistente pelo TreeExplainer — fornece uma análise interpretativa mais completa e robusta: primeiro identifica-se quais variáveis são importantes para o modelo; em seguida, investiga-se como elas influenciam cada previsão individualmente.

Para tornar esse processo mais claro, o fluxo operacional pode ser representado pelo pseudocódigo a seguir, que descreve a integração entre Random Forest e SHAP utilizada neste estudo:

\begin{verbatim}
1. Carregar o dataset original.
2. Separar variáveis preditoras (X) e variável alvo (y).
3. Dividir os dados em treinamento e teste.
4. Treinar o Random Forest com n árvores.
5. Avaliar o desempenho utilizando amostras OOB.
6. Gerar importâncias tradicionais via permutação das variáveis.
7. Inicializar o TreeExplainer a partir do modelo treinado.
8. Calcular valores SHAP para o conjunto de teste.
9. Agregar valores SHAP para obter importância global média.
10. Visualizar importância com summary plots e listas rankeadas.
11. Produzir explicações individuais com gráficos específicos.
\end{verbatim}

O código implementado segue exatamente esse fluxo, permitindo tanto a visualização global da importância média dos atributos — por meio dos gráficos \textit{summary plot} e do ranqueamento numérico — quanto explicações individualizadas das predições, utilizando gráficos como o \textit{force plot}. Adicionalmente, após identificar as oito variáveis mais relevantes segundo a média absoluta dos valores SHAP, realiza-se uma nova visualização focada apenas nesse subconjunto de atributos, facilitando a interpretação e destacando os padrões presentes nas predições do modelo.

Assim, a integração entre Random Forest e SHAP reforça a interpretabilidade do processo de modelagem, oferecendo uma análise robusta, transparente e alinhada às recomendações contemporâneas de Explainable Artificial Intelligence (XAI). Enquanto o Random Forest contribui com estimativas empíricas baseadas em perturbações das variáveis, os valores SHAP complementam essa informação ao explicitar a contribuição individual de cada atributo sobre a predição, garantindo uma visão mais detalhada, confiável e explicativa do comportamento do modelo.


\section{Trabalhos correlatos}\label{correlatos}


\subsection{Análise de Dados na Educação e Estudos sobre Perfil de Estudantes}

A análise de dados na educação emergiu como uma ferramenta transformadora, permitindo uma compreensão profunda e granular do processo de aprendizagem e do comportamento dos estudantes. Com a crescente digitalização do ambiente educacional, a vasta quantidade de dados gerados em plataformas de ensino e sistemas de gerenciamento da aprendizagem oferece um campo fértil para a extração de insights valiosos. Através da aplicação de técnicas avançadas de análise, é possível identificar padrões, prever tendências e, crucialmente, personalizar o ensino, adaptando-o às necessidades individuais de cada aluno. A capacidade de traçar perfis de estudantes, por exemplo, não apenas revela características demográficas, mas também expõe padrões de desempenho, estratégias de aprendizagem e trajetórias acadêmicas, capacitando as instituições a tomarem decisões baseadas em evidências e a otimizar a qualidade da educação.

O trabalho de \cite{guimaraes2023} destaca a importância da análise de dados no contexto educacional como uma ferramenta poderosa para obter insights sobre os estudantes. Segundo os autores, com o avanço da tecnologia e a crescente utilização de plataformas digitais e sistemas de gerenciamento de aprendizagem, tornou-se possível coletar, armazenar e processar grandes volumes de dados educacionais. A análise desses dados permite identificar padrões e tendências, personalizar o ensino de acordo com as necessidades individuais dos estudantes e tomar decisões baseadas em evidências com o objetivo de melhorar a qualidade da educação. Além disso, os autores ressaltam que técnicas como a análise preditiva e o uso de algoritmos de aprendizado de máquina ampliam ainda mais o potencial dessa abordagem.

\cite{oliveira2023} utiliza a análise de dados educacionais para compreender as transformações no perfil dos estudantes de graduação no Brasil entre os anos de 2001 e 2015. A autora realiza uma revisão da literatura e examina diversas bases de dados abertas, como a PNAD, o Censo da Educação Superior, a pesquisa da Andifes/Fonaprace e os resultados do ENADE. Por meio dessas fontes, são evidenciadas mudanças nos perfis dos estudantes quanto à raça/cor, renda e região de origem, revelando um processo de democratização do acesso à educação superior. A análise mostra, por exemplo, o aumento expressivo da participação de estudantes negros e de baixa renda, especialmente nas instituições públicas, como resultado das políticas de inclusão social implementadas no período. Nas considerações finais, a autora destaca a riqueza dos dados disponíveis, especialmente da PNAD, para aprofundar estudos sobre o perfil dos estudantes de forma integrada entre setores público e privado.

O trabalho de \cite{schel2025} investiga as competências de autorregulação da aprendizagem em estudantes de formação em docência. O estudo tem como objetivo identificar diferentes perfis de aprendizagem autorregulada entre esses estudantes. Para alcançar esse objetivo, os autores empregam a análise de perfis latentes (Latent Profile Analysis - LPA), uma técnica estatística multivariada. A LPA é utilizada para agrupar indivíduos em subgrupos (perfis) com base em suas respostas a questionários e testes que avaliam diversas competências de autorregulação da aprendizagem. Ao invés de analisar cada competência isoladamente, a LPA permite identificar combinações de competências que caracterizam grupos distintos de alunos, revelando padrões de pontos fortes e fracos em suas estratégias de aprendizagem. Os resultados da análise de perfis latentes revelaram a existência de quatro perfis distintos de estudantes de formação em docência em relação às suas competências de autorregulação da aprendizagem. Esses perfis são:

\begin{enumerate}[label=\textbf{\arabic*.}]
 \item {Estudantes com altas competências de autorregulação em todos os domínios}: Este grupo demonstra um elevado nível de proficiência em todas as facetas da autorregulação da aprendizagem.

\item {Estudantes com deficiências em estratégias cognitivas}: Este perfil se caracteriza por ter dificuldades específicas no uso de estratégias cognitivas eficazes para a aprendizagem.

 \item {Estudantes com deficiências em estratégias metacognitivas}: Este grupo apresenta lacunas nas suas habilidades de monitoramento e regulação do próprio processo de aprendizagem (metacognição).

 \item {Estudantes com deficiências em estratégias de regulação de recursos}: Este perfil é marcado por dificuldades em gerenciar recursos de aprendizagem, como tempo e ambiente de estudo.
\end{enumerate}

\begin{comment}
Por sua vez, o trabalho "Profile-Based Cluster Evolution Analysis: Identification and Detection of Similar Clusters and Migration Patterns Based on Students' Course-Taking Patterns and Performance" de Priyambada, Mahendrawathi, Yahya e Usagawa (2021) aprofunda-se na análise do comportamento de estudantes universitários através da análise de dados educacionais dinâmicos. O objetivo central é identificar grupos de estudantes com padrões de matrícula em cursos e desempenho acadêmico semelhantes, além de rastrear como esses grupos (clusters) evoluem e como os estudantes migram entre eles ao longo do tempo. Para atingir esses objetivos, os autores utilizam uma abordagem baseada em análise de cluster e análise de evolução de perfil. A análise de cluster é empregada para agrupar estudantes com características semelhantes em termos de suas escolhas de cursos e desempenho, permitindo a criação de "perfis" de estudantes. A inovação do estudo reside na análise da dimensão temporal, investigando como esses clusters mudam ao longo dos semestres ou anos e como os alunos se movem de um perfil para outro. Isso é crucial para entender a dinâmica da jornada acadêmica, identificar pontos de virada ou desafios comuns e, assim, desenvolver intervenções de apoio mais eficazes. Por exemplo, a metodologia de Priyambada et al. (2021) pode revelar que um grupo de alunos inicialmente com alto desempenho pode, em determinado momento, migrar para um perfil de menor desempenho, indicando a necessidade de suporte acadêmico ou aconselhamento. Embora o artigo não liste perfis específicos de estudantes com categorias fixas como o trabalho de Schel e Drechsel, ele foca na metodologia para identificar esses agrupamentos de forma dinâmica, detectando clusters similares ao longo do tempo e rastreando os padrões de migração dos estudantes entre eles com base em suas escolhas de cursos e desempenho. Esta capacidade de monitorar a evolução dos perfis é um avanço significativo para a intervenção proativa e personalizada na trajetória educacional.
\end{comment}

\subsection{Fatores determinantes no interesse e autopercepção em STEM}
A seguir, apresenta-se um fichamento resumido dos principais trabalhos relacionados ao tema deste estudo, conforme a Tabela \ref{tab:tres-colunas}.


\renewcommand{\arraystretch}{1.3}
\begin{longtable}{|p{0.3\textwidth}|p{0.65\textwidth}|}
\caption{Fichamento de trabalhos relacionados}
\label{tab:tres-colunas}\\
\hline
\textbf{Fonte} & \textbf{Descrição} \\
\hline
\endfirsthead
\hline
\textbf{Fonte} & \textbf{Descrição} \\
\hline
\endhead

\parbox[t]{\linewidth}{
    \cite{miller2024} \\
    \cite{master2021} \\
    \cite{mcguire2022} \\
    \cite{master2023}
} & Estereótipos de gênero em torno das habilidades e interesses de meninos e meninas em STEM se formam desde a infância e podem moldar escolhas acadêmicas e de carreira ao longo do tempo. \\
\hline

\parbox[t]{\linewidth}{
    \cite{emran2020} \\
    \cite{liu2024}
} & Pais com formação científica podem expor seus filhos a uma visão mais realista e crítica da ciência, enfatizando suas incertezas, criatividade e natureza interpretativa. \\
\hline

\cite{bohrnstedt2024} & Apesar de meninas demonstrarem menor identidade e autoeficácia matemática em comparação aos meninos, ambos apresentam desempenho matemático semelhante. \\
\hline

\cite{menezes2021} & A presença feminina na Computação ainda é minoritária devido à combinação de estereótipos de gênero, falta de representatividade, desinformação sobre a área e ausência de incentivo familiar e escolar — sendo que iniciativas como oficinas, projetos e parcerias com universidades têm mostrado potencial para reverter esse cenário desde o Ensino Médio. \\
\hline

\cite{zuniga2024} & Estereótipos de gênero em STEM afetam meninas desde o ensino fundamental, influenciam suas aspirações profissionais e são perpetuados por pais, professores e colegas — sendo necessárias intervenções precoces. \\
\hline

\cite{tellhed2023} & Estudantes do ensino fundamental na Suécia apresentam estereótipos implícitos e explícitos que associam tecnologia aos homens e cuidados às mulheres, o que reduz o interesse de meninas por educação tecnológica — especialmente entre aquelas que internalizam mais fortemente esses estereótipos. \\
\hline

\cite{su2023} & Estereótipos sobre perfis de interesse em ciência da computação não refletem a realidade da área e propõem estratégias para tornar STEM mais inclusiva e alinhada à diversidade de interesses, especialmente entre mulheres. \\
\hline

\cite{opps2022} & O estudo revela que meninas do ensino fundamental reproduzem mais estereótipos visuais sobre cientistas da computação do que meninos, e que isso pode impactar negativamente seu senso de pertencimento na área. \\
\hline

\cite{silva2022} & A evasão de mulheres na computação está ligada a estereótipos, discriminação e baixo sentimento de pertencimento, e as soluções atuais ainda são insuficientes. \\
\hline

\cite{elvirazorzo2025} & Mulheres universitárias relatam maiores dificuldades psicossociais, menor autonomia e mais problemas de saúde mental no processo de aprendizagem do que homens.  \\
\hline

\cite{feige2025} & O autoconceito matemático das crianças foi influenciado principalmente pelas expectativas e encorajamento dos pais, especialmente dos pais homens, com efeitos mais fortes sobre os meninos e impacto duradouro sobre as meninas.  \\
\hline

\cite{dulce2022} & A maior exposição de alunas a professoras de STEM no ensino médio está associada a um aumento na matrícula dessas jovens em cursos universitários da área, sugerindo um efeito positivo de modelos femininos na escolha de carreira.  \\
\hline

\cite{mcguire2021} & Interações com educadoras mulheres em espaços informais de ciência aumentam o interesse de meninas por matemática e reduzem estereótipos de que meninos são melhores na área, evidenciando o papel positivo de modelos femininos. \\
\hline

\end{longtable}

