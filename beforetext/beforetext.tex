% ---
% Capa
% ---
\imprimircapa
% ---

% ---
% Folha de rosto
% (o * indica que haverá a ficha bibliográfica)
% ---
\imprimirfolhaderosto*
% ---

% ---
% Inserir a ficha bibliografica
% ---
% http://ficha.bu.ufsc.br/
\begin{fichacatalografica}
	\includepdf{beforetext/ficha-catalografica-tcc.pdf}
\end{fichacatalografica}
% ---

\setlength{\ABNTEXsignwidth}{10cm}

% ---
% Inserir folha de aprovação
% ---
\begin{folhadeaprovacao}
	\OnehalfSpacing
	\centering
	\imprimirautor\\%
	\vspace*{10pt}		
	\textbf{\imprimirtitulo}%
	\ifnotempty{\imprimirsubtitulo}{:~\imprimirsubtitulo}\\%
	%		\vspace*{31.5pt}%3\baselineskip
	\vspace*{\baselineskip}
	%\begin{minipage}{\textwidth}
	% ~do~\imprimirprograma~do~\imprimircentro~da~\imprimirinstituicao~para~a~obtenção~do~título~de~\imprimirformacao.
	Este~\imprimirtipotrabalho~foi julgado adequado para obtenção do título de \imprimirformacao~e aprovado em sua forma final pela banca examinadora. \\
		\vspace*{\baselineskip}
	\imprimirlocal, \imprimirdata. \\
	\vspace*{2\baselineskip}
	\assinatura{\OnehalfSpacing\imprimircoordenador \\ \imprimircoordenadorRotulo~do Curso}
	\vspace*{2\baselineskip}
	\textbf{Banca Examinadora:} \\
	\vspace*{\baselineskip}
	\assinatura{\OnehalfSpacing\imprimirorientador \\ Presidente da Banca}
	%\end{minipage}%
	\vspace*{\baselineskip}
	\assinatura{Prof. X Y Z, Me.\\
	Avaliador \\
	\imprimirinstituicao}

	\vspace*{\baselineskip}
	\assinatura{Prof. X Y Z, Dr.\\
	Avaliador \\
	\imprimirinstituicao}


\end{folhadeaprovacao}
% ---

% ---
% Dedicatória
% ---
%\begin{dedicatoria}
%	\vspace*{\fill}
%	\noindent
%	\begin{adjustwidth*}{}{5.5cm}     
%		Este trabalho é dedicado aos meus colegas de classe e aos meus queridos pais.
%	\end{adjustwidth*}
%\end{dedicatoria}
% ---

% ---
% Agradecimentos
% ---
\begin{agradecimentos}

Agradeço a meu pai, minha mãe, meu cachorro, minha sogra e por último e não menos importante, meu orientador.
\end{agradecimentos}
% ---

% ---
% Epígrafe
% ---
%\begin{epigrafe}
%	\vspace*{\fill}
%	\begin{flushright}
%		\textit{``Texto da Epígrafe.\\
%			Citação relativa ao tema do trabalho.\\
%			É opcional. A epígrafe pode também aparecer\\
%			na abertura de cada seção ou capítulo.\\
%			Deve ser elaborada de acordo com a NBR 10520.''\\
%			(Autor da epígrafe, ano)}
%	\end{flushright}
%\end{epigrafe}
% ---

% ---
% RESUMOS
% ---

% resumo em português
\setlength{\absparsep}{18pt} % ajusta o espaçamento dos parágrafos do resumo
\begin{resumo}
	\SingleSpacing
Este trabalho investiga os fatores que influenciam o interesse e a autopercepção de estudantes universitários em relação às áreas de STEM no município de Salgueiro–PE. A pesquisa analisa como elementos psicossociais, educacionais e experiências de pertencimento moldam percepções distintas entre homens e mulheres. Para isso, são utilizadas técnicas de aprendizado de máquina explicável (Random Forest + SHAP) e métodos de clusterização para identificar perfis recorrentes entre os participantes. Os resultados apontam diferenças importantes na forma como cada gênero internaliza estereótipos, enfrenta experiências de exclusão e desenvolve interesse pela área, evidenciando desafios locais que impactam o acesso e a permanência em STEM no Sertão Central de Pernambuco.

	\textbf{Palavras-chave}: STEM. Gênero. XAI. SHAP. Random Forest. Clusterização. Pertencimento acadêmico.
\end{resumo}

% resumo em inglês
\begin{resumo}[Abstract]
	\SingleSpacing
	\begin{otherlanguage*}{english}


This study investigates the factors that influence university students’ interest and self-perception regarding STEM fields in the municipality of Salgueiro, Brazil. The research examines how psychosocial, educational, and belonging-related experiences shape gender-specific perceptions. Explainable machine learning techniques (Random Forest + SHAP) and clustering methods are applied to identify recurring student profiles. The results reveal meaningful differences in how men and women internalize stereotypes, experience exclusion, and develop interest in STEM, highlighting local challenges that affect access and persistence in these fields within the Sertão Central region of Pernambuco.
		
		\textbf{Keywords}: STEM. Gender. XAI. SHAP. Random Forest. Clustering. Academic belonging.
	\end{otherlanguage*}
\end{resumo}

%% resumo em francês 
%\begin{resumo}[Résumé]
% \begin{otherlanguage*}{french}
%    Il s'agit d'un résumé en français.
% 
%   \textbf{Mots-clés}: latex. abntex. publication de textes.
% \end{otherlanguage*}
%\end{resumo}
%
%% resumo em espanhol
%\begin{resumo}[Resumen]
% \begin{otherlanguage*}{spanish}
%   Este es el resumen en español.
%  
%   \textbf{Palabras clave}: latex. abntex. publicación de textos.
% \end{otherlanguage*}
%\end{resumo}
%% ---

{%hidelinks
	\hypersetup{hidelinks}
	% ---
	% inserir lista de ilustrações
	% ---
	\pdfbookmark[0]{\listfigurename}{lof}
	\listoffigures*
	\cleardoublepage
	% ---
	
	% ---
	% inserir lista de quadros
	% ---
	\pdfbookmark[0]{\listofquadrosname}{loq}
	\listofquadros*
	\cleardoublepage
	% ---
	
	% ---
	% inserir lista de tabelas
	% ---
	\pdfbookmark[0]{\listtablename}{lot}
	\listoftables*
	\cleardoublepage
	% ---
	
	% ---
	% inserir lista de abreviaturas e siglas (devem ser declarados no preambulo)
	% ---
	\imprimirlistadesiglas
	\glsaddall
	% ---
	
	% ---
	% inserir lista de símbolos (devem ser declarados no preambulo)
	% ---
	\imprimirlistadesimbolos
	% ---
	
	% ---
	% inserir o sumario
	% ---
	\pdfbookmark[0]{\contentsname}{toc}
	\tableofcontents*
	\cleardoublepage
	
}%hidelinks
% ---
