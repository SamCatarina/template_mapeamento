
\chapter{Delineamento Metodológico}\label{delineamento}

Este estudo adota uma abordagem quantitativa para investigar o fenômeno da da disparidade de gênero em áreas de STEM. A pesquisa focará na identificação de padrões e na construção de perfis de estudantes com base em suas escolhas. Trata-se de um estudo transversal, o que significa que a coleta de dados ocorrerá em um único período de tempo, oferecendo um panorama das variáveis em questão no momento da pesquisa.

A população-alvo desta pesquisa é composta por estudantes do ensino superior, matriculados em diferentes áreas do conhecimento (incluindo STEM e não-STEM), nas instituições públicas e privadas localizadas na cidade de Salgueiro-PE. A amostra será definida por conveniência, buscando a participação voluntária dos alunos. Todos os participantes serão informados sobre os objetivos do estudo e terão sua anonimidade e confidencialidade, em conformidade com a LGPD.

Os dados serão coletados por meio de um questionário estruturado e autoadministrado, aplicado de forma online via plataforma autoral disponibilizada na web. O instrumento será composto majoritariamente por questões objetivas e fechadas, organizadas em quatro eixos temáticos principais: influências familiares, influências educacionais, fatores psicológicos e características socioeconômicas e demográficas. Essas seções são projetadas para coletar informações detalhadas que abordam os fatores que influenciam as escolhas de curso (sejam eles na área STEM ou não-STEM), além de outras variáveis relevantes para a formação dos perfis dos estudantes. Haverá questões abertas apenas para os casos em que for necessário compreender o conceito ou a forma de pensamento dos participantes sobre determinado assunto.


%pre-proc

Antes da análise, os dados coletados passarão por um processo de pré-processamento, essencial para garantir a qualidade e a consistência dos resultados. Inicialmente, será realizada uma inspeção para identificar e tratar valores ausentes, inconsistentes ou duplicados. As variáveis numéricas serão normalizadas utilizando a técnica de normalização min-max, a fim de padronizar a escala dos dados entre 0 e 1, prevenindo que variáveis com escalas maiores dominem os algoritmos de agrupamento.

Grande parte das respostas às questões fechadas segue uma escala ordinal de intensidade ou frequência (por exemplo, de "não gostava nem me identificava" até "sempre tive muito interesse"), o que permite sua transformação em valores numéricos com base na ordem natural das alternativas. Essa conversão preservará a semântica da resposta, permitindo que tais variáveis sejam utilizadas diretamente nos métodos quantitativos, como o agrupamento.

%analise-estatistica

Após o pré-processamento, será conduzida uma análise estatística descritiva para caracterizar o perfil geral da amostra. Serão calculadas frequências absolutas e relativas, medidas de tendência central (como média e mediana) e de dispersão (como desvio padrão), conforme o tipo de variável. Os dados serão apresentados por meio de tabelas e gráficos descritivos, incluindo gráficos de barras facilitando a visualização e interpretação das principais características dos participantes.

Essa etapa visa identificar tendências gerais da amostra, como proporção de estudantes em áreas STEM e não-STEM, níveis de interesse por disciplinas de exatas e tecnologia. Também serão observadas correlações iniciais entre as variáveis, com o intuito de levantar hipóteses para as etapas posteriores de análise.

%agrupamento

Para a construção dos perfis de estudantes, serão aplicadas técnicas de agrupamento (clustering), com o objetivo de identificar grupos com características semelhantes em relação às variáveis investigadas. A técnica escolhida para essa análise será a clusterização hierárquica, utilizando o método de Ward, que minimiza a variância intra-cluster. A distância entre os pontos será calculada utilizando a métrica de distância euclidiana, adequada para variáveis numéricas. ALém disso, as features selecionadas para a montagem dos clusters serão definidas na fase de análise de importância das variáveis, utilizando o modelo de Random Forest junto com os valores SHAP.

A definição do número ideal de grupos será feita com base no método do cotovelo (elbow method). Após a criação dos clusters, cada grupo será analisado em termos de suas características predominantes, permitindo a descrição dos perfis de estudantes identificados.


